%%%%%%%%%%%%%%%%%%%%%%%%%%%%%%%%%%%%%%%%%%%%%%%%%%%%%%%%%%%%%%%%%%%%%
%%%%%%%%%%%%%%%%%%%%%%%%%%%%%%%%%%%%%%%%%%%%%%%%%%%%%%%%%%%%%%%%%%%%%

\documentclass[useAMS,usenatbib]{mnras}
\usepackage{color}
\usepackage{graphicx}
\usepackage{dcolumn}
\usepackage{bm}
\usepackage{amssymb}
\usepackage{latexsym}
\usepackage[T1]{fontenc}
\usepackage{aecompl} 
\usepackage{amsmath}%
\usepackage{enumitem}

\def\({\left(}
\def\){\right)}
\def\[{\left[}
\def\]{\right]}

%%%%%%%%%%%%%%%%%%%%%%%%%%%%%%%%%%%%%%%%%%%%%%%%%%%%%%%%%%%%%%%%%%%%%
%%%%%%%%%%%%%%%%%%%%%%%%%%%%%%%%%%%%%%%%%%%%%%%%%%%%%%%%%%%%%%%%%%%%%

\begin{document}

\title[$\beta$-Skeleton Analysis]{$\beta$-Skeleton Analysis of the Cosmic Web}

\author[Fang, Forero-Romero, Rossi, Li \& Feng (2018)]
{Feng Fang$^1$, Jaime Forero-Romero$^2$, Graziano Rossi$^3$, Xiao-Dong Li$^1$, Longlong Feng$^1$ \\ \\
$^1$ School of Physics and Astronomy, Sun Yat-Sen University, Guangzhou 510297, P. R. China \\
$^2$ Departamento de F{\'i}sica, Universidad de los Andes, Cra. 1 No. 18A-10 Edificio Ip, CP 111711, Bogot{\'a}, Colombia \\
$^3$ Department of Physics and Astronomy, Sejong University, Seoul, 143-747, Korea}

\pagerange{\pageref{firstpage}--\pageref{lastpage}} \pubyear{2018}
\maketitle
\label{firstpage}


 
%%%%%%%%%%%%%%%%%%%%%%%%%%%%%%%%%%%%%%%%%%%%%%%%%%%%%%%%%%%%%%%%%%%%%
%%%%%%%%%%%%%%%%%%%%%%%%%%%%%%%%%%%%%%%%%%%%%%%%%%%%%%%%%%%%%%%%%%%%%

\section{Introduction}

%0
The spatial distribution of the nearest galaxies on scales of a few
hundreds of Megaparsecs follows a distinct filamentary motif.
This pattern receives the name of cosmic web and has been observed at
different cosmic epochs.  
The search of consistent and stable methods to define this web has
been the subject of continous research for the last 40 years since its
existence was confirmed in early cosmic maps from galaxy redshift
surveys.
See XXX for a recent review.
The cosmic web has also been confirmed both in large galaxy surveys
and also in the dark matter description provided by cosmological
simulations.  

The cosmic web is usually split into four different components: peaks,
sheets, filaments and voids.
There is a great variety of methods to find these components. 
Most of them are focused on the two most prominent features in galaxy
surveys: voids and filaments. 
The voids are regions with sizes on the range of $20-50$ Mpc
practically devoid of galaxies, see XXX for a recent review of
void finder algorithms.
The filaments appear to be the main bridges connecting high
density regions. 
On the largest scales the filament length can be on the order of
$10-100$ Mpc.
See XXX for a recent review of cosmic web finders that are able to
characterize filaments. 

The emergence of the cosmic web can be understood as the interplay of
two conditions. First, the initial gaussian random density field; and
second, its evolution under its own gravity. 
The initial anisotropies in the density field are amplified by gravity
to finally become filaments and voids.  
The structure of the cosmic web is thus expected to encode information
about the underlying cosmological model: type of initial fluctuations,
proportions of different kinds of matter, the expansion history of the
Universe and the rules of gravity.
Voids, for instance, can be used as cosmological probes, their
structure are strongly influence by dark energy.

In this paper we introduce the $\beta$-skeleton as an algorithm to
characterize the cosmic web.
The $\beta$-skeleton concept stems from the fields of computational geometry
and geometric graph theory and has been widely applied in the areas of
image analysis, machine learning, visual perception and pattern
recognition. 
In the context of web finders, the $\beta$-skeleton belongs to a class
of algorithms that, starting from a set of 3D spatial points, builds a
graph describing the degree of connectedness.
In this aspect it is similar to the minimum spanning tree (MST)
algorithm, with the main difference that the resulting graph depends
on the continous $\beta$ parameter;
it is also related to web finders that are designed on the basis of
topological persistence, such as DisPerSE. 


%Kirkpatrick \& Radke (1985) 
%as a scale-invariant variation of the alpha shapes of Edelsbrunner, Kirkpatrick \& Seidel (1983). 
%It has 

%------------------------------------------------------------------------------------------------

% Fixed this by Graziano

This paper is organized as follows. 
In Section 2, we briefly introduce the definition and the basic properties of the $\beta$-skeleton. 
In Section 3, we describe the Big MultiDark Planck (BigMDPL) simulation and the SDSS-III BOSS Data Release 12 (DR12) galaxy sample, which are used later on in the analysis. 
The application of the $\beta$-skeleton statistics is presented in Section 4, 
where we discuss the dependence of the skeleton on the values of $\beta$, 
on the redshift of the various samples, on the redshift-space distortions (RSDs), and on the cosmological volume and Alcock-Paczynski (AP) effects; 
we also graphically illustrate the $\beta$-skeleton constructed from SDSS-III BOSS DR12 galaxies,
and eventually compare the skeletons obtained from observational data and simulated catalogs.  
Finally, we summarize our findings and conclude in Section 5. 


\cite{2017A&A...600A.125C}

\bibliographystyle{mnras}
\bibliography{skel_references} 



\end{document}

