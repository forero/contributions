%%%%%%%%%%%%%%%%%%%%%%%%%%%%%%%%%%%%%%%%%%%%%%%%%%%%%%%%%%%%%%%%%%%%%
%%%%%%%%%%%%%%%%%%%%%%%%%%%%%%%%%%%%%%%%%%%%%%%%%%%%%%%%%%%%%%%%%%%%%

\documentclass[useAMS,usenatbib]{mnras}
\usepackage{color}
\usepackage{graphicx}
\usepackage{dcolumn}
\usepackage{bm}
\usepackage{amssymb}
\usepackage{latexsym}
\usepackage[T1]{fontenc}
\usepackage{aecompl} 
\usepackage{amsmath}%
\usepackage{enumitem}

\def\({\left(}
\def\){\right)}
\def\[{\left[}
\def\]{\right]}

%%%%%%%%%%%%%%%%%%%%%%%%%%%%%%%%%%%%%%%%%%%%%%%%%%%%%%%%%%%%%%%%%%%%%
%%%%%%%%%%%%%%%%%%%%%%%%%%%%%%%%%%%%%%%%%%%%%%%%%%%%%%%%%%%%%%%%%%%%%

\begin{document}

\title[$\beta$-Skeleton Analysis]{$\beta$-Skeleton Analysis of the Cosmic Web}

\author[Fang, Forero-Romero, Rossi, Li \& Feng (2018)]
{Feng Fang$^1$, Jaime Forero-Romero$^2$, Graziano Rossi$^3$, Xiao-Dong Li$^1$, Longlong Feng$^1$ \\ \\
$^1$ School of Physics and Astronomy, Sun Yat-Sen University, Guangzhou 510297, P. R. China \\
$^2$ Departamento de F{\'i}sica, Universidad de los Andes, Cra. 1 No. 18A-10 Edificio Ip, CP 111711, Bogot{\'a}, Colombia \\
$^3$ Department of Physics and Astronomy, Sejong University, Seoul, 143-747, Korea}

\pagerange{\pageref{firstpage}--\pageref{lastpage}} \pubyear{2018}
\maketitle
\label{firstpage}


 
%%%%%%%%%%%%%%%%%%%%%%%%%%%%%%%%%%%%%%%%%%%%%%%%%%%%%%%%%%%%%%%%%%%%%
%%%%%%%%%%%%%%%%%%%%%%%%%%%%%%%%%%%%%%%%%%%%%%%%%%%%%%%%%%%%%%%%%%%%%

\section{Introduction}

The observable universe contains at least 2 trillion galaxies, 
forming the magnificent large-scale structure (LSS) of the Universe. 
LSS surveys in the past three decades have led to a series of important scientific discoveries, 
greatly enriched our knowledge about the Universe.

Driven by gravitational instability, galaxies are organized together and form structures of superclusters, 
sheets, walls and filaments, which are separated by immense voids, 
creating a vast foam-like structure sometimes called the ``cosmic web''. 
A wide range of methods have been developed to identify the cosmic web in simulations or observational catalogues. 
An incomplete list include the friends-of-friends (FOF) method (Huchra \& Geller 1982, Gottloeber 1998), 
SUBFIND (Springel et al. 2001), VOBOZ (Neyrinck, Gnedin \& Hamilton 2005), 
ADAPTAHOP (Aubert, Pichon \& Colombi 2004; Tweed et al. 2009), 
DISPERSE, V-web, T-web, 
and many others 
(for popular void finders, an incomplete list includes the methods developed in Neyrinck (2008), Platen, van de Weygaert \& Jones (2007) 
and Aragon-Calvo et al. (2010a) (see also the references therein), 
while for identifiers of walls and filaments, one can refer to Aragon-Calvo, van de Weygaert \& Jones (2010c), 
Gay et al. (2010), Stoica, Martinez \& Saar (2010), Sousbie et al. (2008a), Stoica et al. 2005, Sousbie et al. 2008b), and so on).

The formation of LSS, driven by the gravitational instability and also influenced by the expansion rate of space-time, 
contains rich information about the nature of the Universe. 
A serious of methods have been designed to extract cosmological information from LSS surveys. 
The baryon acoustic oscillation (BAO) in the early Universe creates unique clustering on the sound horizon scale,
and was used as the ``standard ruler'' to probe the cosmic expansion history. 
Other methods extract information from LSS include redshift space distortion (RSD), Alcock-Paczynski effect (AP), clusters number count, weak lensing (WL), and so on.

Although significant progresses have been made, 
the study of LSS is still far from completion. 
The characterization of Cosmic Web as an open question has been hotly studied for decades, 
yet many debates still exist and it is commonly believed that we are still far from having a mature method 
which can characterize all the key properties of the Cosmic Web while at the same time has a natural mathematical definition 
(cite Void Finder Comparison Project Colberg et al. (2008)). 
For the side of cosmological investigations of LSS, 
the space left for exploration is still vast 
(especially for the beyond 2-point statistics and in the small-scale non-linear clustering region). 
Especially, we are far from understanding the nature and the detailed properties of dark matter and dark energy. 
Driven by the motivation to describe the magnificence of the Universe and the urgent necessity from the cosmological, 
people are continuingly developing and trying ideas to characterize the LSS and extract cosmological information from it.
 
In this paper we made another tentative trial in this direction. 
We study the $\beta$-skeleton statistics and apply it to study the Cosmic Web and cosmology. 
In computational geometry and geometric graph theory, $\beta$-skeletons were first defined by Kirkpatrick \& Radke (1985) 
as a scale-invariant variation of the alpha shapes of Edelsbrunner, Kirkpatrick \& Seidel (1983). 
It has been widely applied in the areas of image analysis, machine learning, visual perception, pattern recognition, and so on. 
We will apply this famous method to the LSS and make preliminary investigations of its properties.

%------------------------------------------------------------------------------------------------

% Fixed this by Graziano

This paper is organized as follows. 
In Section 2, we briefly introduce the definition and the basic properties of the $\beta$-skeleton. 
In Section 3, we describe the Big MultiDark Planck (BigMDPL) simulation and the SDSS-III BOSS Data Release 12 (DR12) galaxy sample, which are used later on in the analysis. 
The application of the $\beta$-skeleton statistics is presented in Section 4, 
where we discuss the dependence of the skeleton on the values of $\beta$, 
on the redshift of the various samples, on the redshift-space distortions (RSDs), and on the cosmological volume and Alcock-Paczynski (AP) effects; 
we also graphically illustrate the $\beta$-skeleton constructed from SDSS-III BOSS DR12 galaxies,
and eventually compare the skeletons obtained from observational data and simulated catalogs.  
Finally, we summarize our findings and conclude in Section 5. 


\cite{2017A&A...600A.125C}

\bibliographystyle{mnras}
\bibliography{skel_references} 



\end{document}

